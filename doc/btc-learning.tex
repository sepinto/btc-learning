\documentclass[12pt]{article}
\usepackage{fullpage,graphicx,psfrag,amsmath,amsfonts,verbatim}
\usepackage[small,bf]{caption}

\input defs.tex

\bibliographystyle{alpha}

\title{CS 229 Project}
\author{Robert Konrad \& Stephen Pinto}

\begin{document}
\maketitle

\section{Motivation} % (fold)
\label{sec:motivation}
We are predicting how long a UTXO will stay unspent. Services in the business of providing liquidity for their customers would benefit from knowing how long a customer might leave some amount of BTC unspent.

\section{Initial Exploration}


\section{Clustering}

\section{Classification}

\section{EM for Laplacian Mixture}
Consider a model for the set of $m$ unlabelled samples $\left\{ x^{(1)}, \dots, x^{(m)} \right\}$. Underlying each sample is a latent random variable $z^{(i)}$ with 
\begin{equation}
\label{z}
z^{(i)} \sim \text{Multinomial}_k(\phi)
\end{equation}
and
\begin{equation}
\label{z}
x^{(i)} | z^{(i)} = j \sim \text{Laplace}(\mu_j, b_j > 0) = \frac{1}{2 b_j} \exp \left( - \frac{| x^{(i)} - \mu_j |}{b_j} \right).
\end{equation}
As is standard with the EM derivation, we are trying to maximize the log likelihood function
\[
\ell \left( \theta \right) = \sum_i \log \left[ \sum_{z^{(i)}} Q_i(z^{(i)}) \frac{p(x^{(i)}, z^{(i)} ; \theta)}{Q_i(z^{(i)})} \right]
\]
with parameters $\theta = [\phi\ \mu\ b]^T$ by using a lower bound from Jensen's inequality
\[
\ell \left( \theta \right) \geq \sum_i \sum_{z^{(i)}} Q_i(z^{(i)}) \log  \frac{p(x^{(i)}, z^{(i)} ; \theta)}{Q_i(z^{(i)})}.
\]
Given some initial $\theta$, the E step creates the tightest bound possible by setting
\[
Q_i(z^{(i)}) = p(z^{(i)} | x^{(i)} ; \theta) = \omega_j^{(i)} = \frac{p(x^{(i)} | z^{(i)} ; \mu, b) p(z^{(i)} ; \phi)}{\sum_{l} p(x^{(i)} | z^{(i)} = l ; \mu_l, b_l) \phi_l}.
\]
This is straightforward to calculate. With the function $Q_i(z^{(i)})$ set, the M step is
\[
\theta := \argmax_{\theta} \sum_i \sum_{z^{(i)}} Q_i(z^{(i)}) \log  \frac{p(x^{(i)}, z^{(i)} ; \theta)}{Q_i(z^{(i)})}
\]
which, immediately removing terms that don't depend on $\theta$, is equivalent to
\[
\theta := \argmax_{\theta} \sum_i \sum_{z^{(i)}} Q_i(z^{(i)}) \log  \left[ p(x^{(i)} | z^{(i)} ; \mu, b) p( z^{(i)} ; \phi) \right].
\]
As derived in the book, we easily get
\begin{equation}
\label{phiUpdate}
\phi_j := \frac{1}{m} \sum_{i=1}^{m} \omega_j^{(i)}.
\end{equation}
Plugging in our Laplace PDF in order to find the update rules for $\mu$ and $b$, we get
\[
\mu, b := \argmin_{\mu, b} \sum_{i=1}^m \sum_{j=1}^k \omega_j^{(i)} \left( \frac{|x^{(i)} - \mu_j|}{b_j} + \log b_j \right)
\]
Note that this expression is not convex in b. $\log b_j$ is concave and $1/b_j$ is convex so the sum of the two is neither. Multiplying by 2, the equivalent expression
\[
\mu, b := \argmin_{\mu, b} \sum_{i=1}^m \sum_{j=1}^k \omega_j^{(i)} \left(  \frac{2 |x^{(i)} - \mu_j|}{s_j} + \log b_j^2 \right)
\]
solves this with a change of variables $s = -b$, becoming
\[
\mu, s := \argmin_{\mu, s} \sum_{i=1}^m \sum_{j=1}^k \omega_j^{(i)} \left(  \frac{- 2 |x^{(i)} - \mu_j|}{s_j} + \log s_j^2 \right).
\]
Ignoring the constraint $s \prec 0$ for one moment and just setting the derivative of this expression with respect to $s_j$ to zero gives
\[
\frac{\partial \left[\ \cdot\ \right]}{\partial s_j} = \sum_i \omega_j^{(i)} \left( \frac{1}{s_j} + \frac{|x^{(i)} - \mu_l|}{s_j^2} \right) = 0.
\]
Solving for $s_j$ yields
\[
s_j := - \frac{\sum_i \omega_j^{(i)} | x^{(i)} - \mu_j|}{\sum_i \omega_j^{(i)}}
\]
and, equivalently,
\begin{equation}
\label{sUpdate}
b_j := \frac{\sum_i \omega_j^{(i)} | x^{(i)} - \mu_j|}{\sum_i \omega_j^{(i)}}.
\end{equation}
Note that this value will always be less than zero, so our constraint is taken care of.

With $\phi$ taken care of and an expression for $b$ in hand, that leaves just 
\[
\mu := \argmin_\mu \sum_i \sum_j \frac{\omega_j^{(i)}}{b_j} | x^{(i)} - \mu_j |.
\]
Clearly the optimal $\mu_j$ does not depend on $b_j$, so we can consider the simpler expression
\[
\mu := \argmin_\mu \sum_i \sum_j \omega_j^{(i)} | x^{(i)} - \mu_j |.
\]
As with all the other parameters so far, this expression is separable with respect to elements of $\mu$. That is, we can say
\[
\mu_j := \argmin_{\mu_j} \sum_i \omega_j^{(i)} | x^{(i)} - \mu_j |
\]
Setting the partial to zero gives us
\[
\sum_i \omega_j^{(i)} \ones \left\{ \mu_j > x^{(i)} \right\} = \sum_i \omega_j^{(i)} \ones \left\{ \mu_j < x^{(i)} \right\}.
\]
 

\textbf{THIS IS INCORRECT.} See notes below from Junjie.

Instead of setting the derivative to 0, you can use the sign of the derivative to determine whether the function is decreasing or increasing. For instance, you can show that f is decreasing if
\[
\sum \omega_j^{(i)} \ones\{\mu < x^{(i)}\} > \sum \omega_j^{(i)} \ones \{ \mu > x^{(i)} \},
\]
and increasing if the inequality is flipped. In fact, you can show that the function is convex so that f' is monotone and hence you can pick $\mu^\star$ to be the smallest point such that $f'(\mu)$ changes its sign (i.e. from negative to positive). Observing that
\[
\ones \{ \mu < x^{(i)} \} = 1- \ones \{ \mu \geq x^{(i)} \}, 
\]
and if the data is sorted such that $x^{(1)} \leq \dots \leq x^{(m)}$, we can set $\mu^\star = x^{(k)}$, where k is the smallest index such that 
\[
\sum_{i=1}^{k-1} \omega_j^{(i)} < 0.5 \sum_{i=1}^n \omega_j^{(i)},
\]
(i.e. the function is decreasing on the left side of $\mu^\star$), and
\[
\sum_{i=1}^{k} \omega_j^{(i)} 
\geq 0.5 \sum_{i=1}^n \omega_j^{(i)},
\]
(i.e. the function is nondecreasing on the right side of $\mu^\star$).



\bibliography{template}

\end{document}